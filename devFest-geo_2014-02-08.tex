% Created 2014-02-02 Sun 15:56
\documentclass[sans,aspectratio=169,presentation,bigger,fleqn]{beamer}
\usepackage[utf8]{inputenc}
\usepackage[T1]{fontenc}
\usepackage{fixltx2e}
\usepackage{graphicx}
\usepackage{longtable}
\usepackage{float}
\usepackage{wrapfig}
\usepackage{rotating}
\usepackage[normalem]{ulem}
\usepackage{amsmath}
\usepackage{textcomp}
\usepackage{marvosym}
\usepackage{wasysym}
\usepackage{amssymb}
\usepackage{hyperref}
\tolerance=1000
\usepackage{setspace}
\setstretch{1.3}
\usepackage{booktabs}
\hypersetup{colorlinks=true,linkcolor=blue,urlcolor=blue}
%\usetheme{naked}
\usepackage{lmodern}
\usetheme[alternativetitlepage=true,titleline=true]{Torino}
\usecolortheme{freewilly}
\usetheme{default}
\author{John Henderson}
\date{08 February 2014}
\title{Working with geo-spatial data in R}
\hypersetup{
  pdfkeywords={},
  pdfsubject={},
  pdfcreator={Emacs 24.3.1 (Org mode 8.2.5h)}}
\begin{document}

\maketitle


\begin{frame}[fragile,label=sec-1]{Intro}
 \begin{itemize}
\item Brief overview of geo-spatial miscellany in R
\item Key libraries
\begin{itemize}
\item Getting coordinates/working with Google: \texttt{ggmap}
\item Great circle paths: \texttt{geosphere}
\item Larger scale maps: \texttt{maps} (comes with R)
\item Projection conversion/KML generation: \texttt{rgdal}, \texttt{maptools}, and \texttt{sp}
\item Plotting: \texttt{ggplot2}
\end{itemize}
\end{itemize}
\pause

\begin{itemize}
\item This is a hobby and I'm \emph{not} a professional\ldots{}
\end{itemize}
\end{frame}
\begin{frame}[fragile,label=sec-2]{Using \texttt{ggmap} to get lat/lon coordinates}
 \begin{itemize}
\item Just type what you wold type into Google Maps
\end{itemize}


\scriptsize

\begin{verbatim}
  # install.packages("ggmap")
  > library(ggmap)
  > geocode("St. Paul, MN")
\end{verbatim}

\begin{verbatim}
        lon     lat
1 -93.08996 44.9537
\end{verbatim}


\begin{verbatim}
  > geocode("2115 Summit Ave., St. Paul, MN")
\end{verbatim}

\begin{verbatim}
        lon      lat
1 -93.18971 44.94412
\end{verbatim}

\begin{verbatim}
  > geocode("University of St. Thomas, MN")
\end{verbatim}

\begin{verbatim}
        lon      lat
1 -93.18975 44.94192
\end{verbatim}

\normalsize
\end{frame}

\begin{frame}[label=sec-3]{Grabbing maps}
Methods
\begin{itemize}
\item Lat/lon + zoom level
\item Bounding box
\end{itemize}

Sources
\begin{itemize}
\item Google
\item Stamen
\item Cloudmade
\item OpenStreetMap
\end{itemize}
\end{frame}
\begin{frame}[fragile,label=sec-4]{Using lat/lon + zoom}
 \begin{verbatim}
loc <- geocode("2115 Summit Ave, St. Paul, MN")
ust <- get_map(location = c(lon = loc$lon, lat = loc$lat),
                zoom = 15, source = "google",
                maptype = "hybrid", crop = T)

ggmap(ust)
\end{verbatim}
\end{frame}
\begin{frame}[label=sec-5]{Using lat/lon + zoom}
\begin{center}
\includegraphics[height=6.5cm]{./img/ust-coords-zoom.pdf}
\end{center}
\end{frame}
\begin{frame}[fragile,label=sec-6]{Using lat/lon + bounding box}
 \begin{verbatim}
loc <- geocode("2115 Summit Ave, St. Paul, MN")

box <- c(left = loc$lon - 0.04, bottom = loc$lat - 0.02,
         right = loc$lon + 0.04, top = loc$lat + 0.02)

ust_box <- get_map(location = box, source = "stamen",
                   maptype = "watercolor", crop = T)

ggmap(ust_box)
\end{verbatim}
\end{frame}
\begin{frame}[label=sec-7]{Using lat/lon + bounding box}
\begin{center}
\includegraphics[height=6.5cm]{./img/ust-coords-box.pdf}
\end{center}
\end{frame}
\begin{frame}[fragile,label=sec-8]{Overplotting ggmaps}
 \begin{itemize}
\item \texttt{ggmap()} is an addition to \texttt{ggplot2} functionality
\item Thus, layering standard \texttt{ggplot2} graphics on top of maps is easy!
\end{itemize}

\scriptsize
\begin{verbatim}
locs <- data.frame(names = c("st. paul", "minneapolis"))
locs <- cbind(locs, geocode(as.character(locs$names)))

mid <- get_map(location = c(lon = mean(locs$lon), lat = mean(locs$lat)),
               zoom = 10, source = "stamen", maptype = "toner", crop = T)

p <- ggmap(mid)
p <- p + geom_point(aes(x = lon, y = lat, colour = factor(names)),
                    dat = locs, size = 6)
p <- p + scale_colour_discrete("Location")
p
\end{verbatim}

\normalsize
\end{frame}
\begin{frame}[label=sec-9]{Overplotting ggmaps}
\begin{center}
\includegraphics[height=6.5cm]{./img/ggmap-points.pdf}
\end{center}
\end{frame}
\begin{frame}[fragile,label=sec-10]{Working with world/contry maps}
 \begin{itemize}
\item \texttt{get\_map()} limited by allowable zoom levels (e.g. \texttt{0 < zoom < 21})
\item To plot larger scale data, the \texttt{maps} package is probably preferred
\item Various maps available; see \href{http://cran.r-project.org/web/packages/maps/maps.pdf}{documentation} for details
\end{itemize}

\scriptsize
\begin{verbatim}
library(maps)
world <- map_data("world")
head(map, 5)
\end{verbatim}

\begin{verbatim}
       long      lat group order region subregion
1 -133.3664 58.42416     1     1 Canada      <NA>
2 -132.2681 57.16308     1     2 Canada      <NA>
3 -132.0498 56.98610     1     3 Canada      <NA>
4 -131.8797 56.74001     1     4 Canada      <NA>
5 -130.2492 56.09945     1     5 Canada      <NA>
\end{verbatim}
\end{frame}
\begin{frame}[fragile,label=sec-11]{World map example}
 \scriptsize
\begin{verbatim}
# pay attention to column names (lon vs. long!)
p <- ggplot(world, aes(x = long, y = lat, group = group))
p <- p + geom_polygon(colour = "white")
p
\end{verbatim}

\begin{center}
\includegraphics[height=4.5cm]{./img/world.pdf}
\end{center}

\normalsize
\end{frame}
\begin{frame}[fragile,label=sec-12]{United States example}
 \scriptsize
\begin{verbatim}
usa <- map_data("state")
p <- ggplot(usa, aes(x = long, y = lat, group = group))
p <- p + geom_polygon(colour = "white")
p
\end{verbatim}

\begin{center}
\includegraphics[height=4.5cm]{./img/usa.pdf}
\end{center}

\normalsize
\end{frame}
\begin{frame}[fragile,label=sec-13]{Subsetting areas}
 \scriptsize
\begin{verbatim}
states <- c("minnesota", "wisconsin", "illinois", "indiana",
            "iowa", "missouri", "michigan")

states_map <- map_data("state")

states_map <- states_map[states_map$region %in% states, ]

p <- ggplot(states_map, aes(x = long, y = lat, group = group))
p <- p + geom_polygon(colour = "white")
p
\end{verbatim}
\end{frame}
\begin{frame}[label=sec-14]{Subsetting areas}
\begin{center}
\includegraphics[height=6cm]{./img/some-states.pdf}
\end{center}
\end{frame}
\begin{frame}[label=sec-15]{A hobby project}
\begin{itemize}
\item Internal talks at 3M typically given live to US audience; recorded for int'l
\item Organized two "reverse talks" to reach a wider global audience
\item Attendance records shattered
\item Wanted to visualize impact/reach!
\end{itemize}
\end{frame}
\begin{frame}[label=sec-16]{The inspiration}
\begin{center}
\includegraphics[height=6cm]{./img/facebook-map-lo.png}
\end{center}
\end{frame}
\begin{frame}[fragile,label=sec-17]{Great circle basics}
 \begin{itemize}
\item \texttt{gcIntermediate()} function from  \texttt{geosphere} package
\end{itemize}

\scriptsize
\begin{verbatim}
library(geosphere)
arc <- gcIntermediate(c(lon_1, lat_1), c(lon_2, lat_2),
                      n = steps, addStartEnd = T)
\end{verbatim}
\end{frame}
\begin{frame}[fragile,label=sec-18]{Crossing the Date Line}
 \scriptsize
\begin{verbatim}
# draw great circles from St. Paul to everywhere else
gcircles <- lapply(1:nrow(end), function(i) {
  temp <- gcIntermediate(start[, c("lon", "lat")], end[i, c("lon", "lat")],
                         n = 50, addStartEnd = T, breakAtDateLine = T)

  # if temp is a list, rbind and return single data.frame
  if(is.list(temp) == T) {
    ids <- c(rep(paste0("i", i), nrow(temp[[1]])),
             rep(paste0("j", i), nrow(temp[[2]])))
    temp <- as.data.frame(rbind(temp[[1]], temp[[2]]))
    temp$id <- ids
  }
  
  ...
})
\end{verbatim}
\normalsize
\end{frame}
\begin{frame}[fragile,label=sec-19]{The plot}
 \scriptsize
\begin{verbatim}
p <- ggplot()
p <- p + geom_polygon(aes(x = long, y = lat, group = group),               # world map
                      data = world, colour = "gray10", fill = "gray95")
p <- p + geom_line(aes(x = lon, y = lat, group = id),                      # great circles
                   dat = gcircles, lwd = 0.4, alpha = 0.5)
p <- p + geom_point(aes(x = lon, y = lat, size = sqrt(total/pi)),          # points
                    dat = talks_agg, colour = "#555599")
p <- p + scale_size("Total participants\n(both events)",                   # adjust legend
                    limits = c(0, max(sqrt(talks_agg$total / pi)) + 1),
                    breaks = sqrt(c(10, 50, 100) / pi),
                    labels = c(10, 50, 100), range = c(1, 10))
p <- p + theme_bw()                                                        # bw theme
p <- p + theme(axis.text = element_blank(), axis.title = element_blank(),  # tweak
               axis.ticks = element_blank(), panel.grid = element_blank(),
               legend.position = c(0.092, 0.15))
\end{verbatim}
\normalsize
\end{frame}

\begin{frame}[label=sec-20]{The result}
\normalsize


\begin{center}
\includegraphics[height=6.5cm]{./img/great-circles.pdf}
\end{center}
\end{frame}
\begin{frame}[label=sec-21]{Infographic}
\begin{center}
\includegraphics[height=6.5cm]{./img/infographic-flattened-lo.jpg}
\end{center}
\end{frame}
\begin{frame}[label=sec-22]{Getting some GPS data}
\begin{center}
\href{https://play.google.com/store/apps/details?id=com.fivasim.androsensor&hl=en}{\uline{AndroSensor}}
\end{center}

\begin{columns}
\begin{column}{0.45\textwidth}
\begin{center}
\includegraphics[height=5cm]{./img/andro-main.png}
\end{center}
\end{column}
\begin{column}{0.45\textwidth}
\begin{center}
\includegraphics[height=5cm]{./img/andro-sensors.png}
\end{center}
\end{column}
\end{columns}
\end{frame}
\begin{frame}[fragile,label=sec-23]{Reading/cleaning the data}
 \scriptsize
\begin{verbatim}
gps <- read.csv("./data/gps-data.csv", sep = ";")

# reduce to columns of interest
gps <- gps[, c(10, 11, 13, 18)]

# give the data sensical names
names(gps) <- c("lat", "lon", "speed", "time")
head(gps, 5)
\end{verbatim}

\begin{verbatim}
       lat       lon speed time
1 44.92633 -93.09771    NA    5
2 44.92633 -93.09771    NA  505
3 44.92633 -93.09771    NA 1011
4 44.92633 -93.09771    NA 1532
5 44.92626 -93.09747     0 2032
\end{verbatim}

\normalsize
\end{frame}
\begin{frame}[fragile,label=sec-24]{Grab a background map}
 \scriptsize

\begin{verbatim}
box <- c(left = min(gps$lon), bottom = min(gps$lat),
         right = max(gps$lon), top = max(gps$lat))

gps_map <- get_map(location = box, source = "stamen",
                   maptype = "terrain", crop = T)
\end{verbatim}

\begin{center}
\includegraphics[height=4cm]{./img/gps-map.pdf}
\end{center}
\end{frame}

\begin{frame}[fragile,label=sec-25]{Overplot with speed}
 \scriptsize

\begin{verbatim}
p <- ggmap(gps_map) + geom_point(aes(x = lon, y = lat, colour = speed),
                                 gps, size = 3)
p <- p + scale_colour_continuous(low = "black", high = "red", na.value = NA)
\end{verbatim}

\begin{center}
\includegraphics[height=5cm]{./img/gps-map-over.pdf}
\end{center}

\normalsize
\end{frame}
% Emacs 24.3.1 (Org mode 8.2.5h)
\end{document}
